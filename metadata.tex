
\newcommand{\pubtitle}{The Core language of Juvix}

\newcommand{\pubauthA}{Lukasz Czajka}
\newcommand{\pubaffilA}{a}
% \newcommand{\orcidA}{0000-0001-5477-1503}
\newcommand{\authemailA}{lukasz@heliax.dev}
% \newcommand{\eqcontribA}{}

% \newcommand{\pubauthB}{Poison Ivy}
% \newcommand{\pubaffilB}{a}
% \newcommand{\orcidB}{0000-0001-0000-0000}
% \newcommand{\authemailB}{poison@heliax.dev}
% \newcommand{\eqcontribB}{}

% \newcommand{\pubauthC}{Last Author}
% \newcommand{\pubaffilC}{a}
% \newcommand{\orcidC}{0000-0001-5477-1503}
% \newcommand{\authemailC}{mail@someinstitute.com}

% Institutions/Affiliations
\newcommand{\pubaddrA}{Heliax AG}

% Corresponding author mail
\newcommand{\pubemail}{\authemailA}

\newcommand{\pubabstract}{%
This report describes JuvixCore -- a minimalistic intermediate functional
language to which Juvix desugars. We provide a precise and abstract
specification of JuvixCore's syntax, evaluation semantics, and optional
type system. We comment on the relationship between this specification and the actual implementation. We also explain the role JuvixCore plays in the Juvix compilation pipeline. Finally, we compare the language
features available in JuvixCore with those in Juvix and other popular
functional languages.
}

% Description of the SI file, placed as a footnote
% \newcommand{\pubSI}{Electronic Supplementary Information (ESI) available:
% one PDF file with all referenced supporting information.}

% Any keywords to be displayed under the abstract
\keywords{ 
Juvix\sep
Language specification\sep
Functional programming\sep
Compilers\sep
Lambda Calculus\sep
}
% Supplementary space between title/abstract and text, if needed
% \newcommand{\pubVadj}{0pt}

% ! DO NOT REMOVE OR MODIFY !
\input{templates/ART/aux-preamble.tex}
% The preprint DOI to be used as an link in the paper

\pubdoi{10.5281/zenodo.8297159}
\history{(Received July 31, 2023; Published: August 29, 2023; Version: \today)}